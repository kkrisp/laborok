\documentclass{article}
\usepackage[T1]{fontenc}
\usepackage[utf8]{inputenc}
\usepackage[magyar]{babel}

\usepackage{hyperref}
\usepackage{float}

\usepackage{anysize}
%marginsize{left }{right}{top  }{bottom}
\marginsize{2.0cm}{2.0cm}{1.0cm}{2.0cm}

\usepackage{fancyhdr}
\pagestyle{fancy}
\fancyhf{}
\rhead{Modern fizika laboratórium}
\lhead{Granuális anyagok}
\fancyfoot[c]{\large\thepage.oldal}
\renewcommand{\headrulewidth}{0pt}

\begin{document}
	\section{Bevezetés}
	Granuláris anyagoknak a nagy számú ($10^4 - 10^6$ db) makroszkópikus részecskékből( $10 \mu m - 10 m$ átmérőjű)  álló rendszereket nevezzük.  A granuláris részecskékre ható erőhatások közül a gravitációs erő, két részecske összenyomódásakor fellépő taszítóerő és az érintkezési pontokon fellépő surlódási erő tartozik a legjelentősebbek. A granuláris anyag részecskéinek az átlagos helyzeti energiájához képest az egy szabadsági fokra jutó $k_BT$ ter mikus energia elhanyagolható. Emiatt a hőmérsékletnek az átlagoló befolyása elvész ezeknél a komplex rendszereknél. Ennek köbetkeztében nem alakul a rendszeren belül termikus egyensúly. Ez azt jelenti, hogy külső megzavarás nélkül a rendszer bármely metastabil állapota végtelen sok ideig fennmaradhat. Keveredés, homogén eloszlások kialakulása helyett rendeződés, szegregáció, komplex struktúrák alakulnak ki.
	
	\section{A mérés célja}
	
	A mérés célja a granuláris anyagokban fellépő nyomás mélységfüggésének a vizsgálata a \textbf{Janssen modellel}, majd a rendszeren belüli inhomogén erőeloszlásoknak erőláncoknak a vizsgálata a \textbf{q-modell} alkalmazasávál.
	
	\section{A mérés elve}
	
	A nyomás mélységfüggésenek a leírása granuláris oszlopokban különbözik a folyadékoszlopokban létrejövő nyomás leírásától. Granuláris anyagok esetében a folyadékoknál használt $P(z)=\rho gz$ összefüggés nem áll fent. Azért, hogy meghatározzuk a granuláris oszlopok által keletkeztetett nyomást a Jennsen-modell legfőbb feltevéseit tekintjük.
	Tekintsünk egy $R$ sugarú, $\rho$ átlagos sűrűségű hengeres edényt, amelyet granuláris anyaggal megtöltünk.
	A Jennsen-modell értelmében:
	
	A függőleges nyomás nagysága csak a mélységfügtől függ
	
	$$P(x,y,z)=P(z)$$
	
	A vízszintes irányban mérhető nyomás arányos a függőleges nyomással
	$$P_hor=KP( z )$$
	
	
	Az üveghenger falainal fellépő tapadási surlódási erők felfelé mutatnak és mind a maximális értékeiket veszik fel
	
	$$dF_frict=\mu KP( z )2\pi Rdz$$
	
	
	ahol $\mu$ a henger üvegfala és a granuláris anyag közti statikus súrlódási együttható
	Továbbá a Jennsen-modell szerint az anyag minden $dz$ vastagságú és $S=\pi R^2$ felületű vízszintes szeletének egyensúlyban kell lennie. Tehát, mivel ezekre a szeletekre hat a gravitációs erő, a fölötte és alatta mérhető nyomás különbségéből adódó és az üvegfalaknál fellépő súrlódási erő, ezért a Newton-egyenlet
	
	$$\rho gSdz-\frac{dP( z )}{dz} Sdz-dF_{frict}=0$$
	
	
	A Jennsen-modell egyik feltevéséből kapott $dF_{frict}$-et behelyettesítjük a Newton-egyenletbe, majd átrendezve azt kapjuk, hogy 
	$${dP( z )}+ \frac{1}{\lambda P}=\rho g$$
	
	
	
	ahol
	
	
	$$\lambda = \frac{R}{2 \mu K}$$
	A $P(0)=0$ kezdeti feltétellet a differenciálegyenlet megoldása 
	$$P(z) 7 \lambda \rho g \left(1 - e^{\frac{z}{\lambda}}\right)$$
	
	
	Tehát tetszőlegesen nagy z-re a nyomás nem divergál, hanem $\lambda$ karakterisztikus távolságon exponenciálisan megy telítésbe. A konkrét feladatra a differenciálegyenlet megoldása
	
	$$P(z) = m_\infty \left( 1-e^{\frac{z}{m_\infty}}\right)$$
	
%	A mérés során az $m_\infty$ értéket kell meghatároznunk különboző töltési módszerek és granuláris anyagok esetén.
	
	\section{A mérés során felhasznált mérési eszközök}
	\begin{itemize}
		\item Különböző granuláris anyagok: köles, műanyag- és üveggolyók
		\item Műanyagpoharak
		\item Merőkanál
		\item Üveghenger
		\item Elektronikus táramérleg
		\item Talpas fémhenger dugattyúval
		\item Kartonpapír
		\item Fénymásoló A4 formátumú papír
		\item Indigó
	\end{itemize}
	
	\section{A mérés menete}
	\subsection{title}
\end{document}
